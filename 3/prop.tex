\section{Some Properties of Probability Measure and More Counting}
\subsection{Elementary Results}
Let $(\Omega,\mathscr F,\mathbb P)$ be a probability space.
\begin{proposition}[Countable Subadditivity]
    Let $(A_n)_{n\in\mathbb N}\in\mathscr F$, then
    $$\mathbb P\left( \bigcup_{n\in\mathbb N}A_n \right)\le\sum_{n=1}^\infty \mathbb P(A_n)$$
\end{proposition}
\begin{proof}
    Consider $B_1=A_1$ and $B_n=A_n\setminus (A_1\cup A_2\cup\cdots\cup A_{n-1})$.
    The result follows.
\end{proof}
$\mathbb P$ is upward continuous.
\begin{proposition}
    Suppose $(A_n)_n\in\mathbb N\in\mathscr F$ is increasing, then
    $$\mathbb P(A_n)\to \mathbb P\left( \bigcup_{n\in\mathbb N} A_n\right)$$
    as $n\to\infty$.
\end{proposition}
\begin{proof}
    Consider $B_1=A_1$ and $B_n=A_n\setminus (A_1\cup A_2\cup\cdots\cup A_{n-1})$.
    Note that $\mathbb P(A_n)=\mathbb P(B_1)+\cdots+\mathbb P(B_n)$, so
    $$\mathbb P(A_n)=\sum_{k=1}^n\mathbb P(B_k)=\mathbb P\left(\bigcup_{k=1}^nB_n\right)=\mathbb P\left( \bigcup_{k=1}^nA_n \right)\to\mathbb P\left( \bigcup_{n\in\mathbb N} A_n\right)$$
    As desired.
\end{proof}
By taking complement, we know that $\mathbb P$ is also downward continuous.
\begin{proposition}[Inclusion-Exclusion Principle]
    Let $A_1,A_2,\ldots,A_n$ be events, then we have
    $$\mathbb P(A_1\cup\cdots\cup A_n)=\sum_{k=1}^n(-1)^{k+1}\sum_{1\le i_1<\cdots<i_k\le n}\mathbb P(A_{i_1}\cap\cdots\cap A_{i_k})$$
\end{proposition}
\begin{proof}
    Induction.
\end{proof}
For the probability on finite sample space with equally likely outcomes, the principle reduced to the known inclusion-exclusion principle on finite sets.
\begin{corollary}
    Let $A_1,A_2,\ldots,A_n$ be sets, then we have
    $$|A_1\cup\cdots\cup A_n|=\sum_{k=1}^n(-1)^{k+1}\sum_{1\le i_1<\cdots<i_k\le n}|A_{i_1}\cap\cdots\cap A_{i_k}|$$
\end{corollary}
\begin{proof}
    Follows directly.
\end{proof}
\begin{proposition}[Bonferroni Inequality]
    Truncating the inclusion-exclusion formula after the $n^{th}$ term gives an overestimate if $n$ is odd and underestimate if $n$ is even.
\end{proposition}
\begin{proof}
    Trivial.
\end{proof}
\subsection{Counting by Inclusion-Exclusion}
Inclusion-Exclusion allows us to do some more counting.\\
We want to count the number of surjections $\{1,2,\ldots,n\}\to\{1,2,\ldots,m\}$ for $m\le n$.
Take $\Omega=\{f:\{1,2,\ldots,n\}\to\{1,2,\ldots,m\}\}$ and $A$ be the set of surjections.
Let $A_i=\{f\in\Omega:i\notin \operatorname{Im}f\}$, so $A=A_1^c\cap\cdots\cap A_m^c=(A_1\cup\cdots\cup A_m)^c$.
Now since we have $|A_{i_1}\cap\cdots\cap A_{i_k}|=(m-k)^n$, hence by Inclusion-Exclusion, we have
$$|A|=\sum_{k=0}^m(-1)^k\binom{m}{k}(m-k)^n$$
(Note that a complement has been taken.)\\
We now want to count the number of derangements of $\{1,2,\ldots,n\}$.
Let $\Omega=S_n$ and $A$ be the set of derangements.
Let $A_i=\{\sigma\in\Omega:\sigma(i)=i\}$, so again $A=(A_1\cup\cdots\cup A_n)^c$.
But we have $|A_{i_1}\cap\cdots\cap A_{i_k}|=(n-k)!$, so assume an equally likely probability measure $\mathbb P$, we then have
$$\mathbb P(A^c)=\sum_{k=1}^n(-1)^{k+1}\frac{n}{k}\frac{(n-k)!}{n!}\implies \mathbb P(A)=\sum_{k=0}^n(-1)^k\frac{1}{k!}\to \frac{1}{e}$$
So asymptopically $|A|\sim n!/e$.
\subsection{Independence}
\begin{definition}
    $A,B\in\mathscr F$ are said to be independent if $\mathbb P(A)\mathbb P(B)=\mathbb (A\cap B)$.\\
    A countable sequence $(A_n)_{n\in\mathbb N}$ of events is independent if $\forall k\ge 2$ and for any set of indices $i_1,i_2,\ldots,i_k$ we have
    $$\mathbb P(A_{i_1}\cap\cdots\cap A_{i_k})=\mathbb P(A_{i_1})\cdots\mathbb P(A_{i_k})$$
\end{definition}
\begin{remark}
    1. Pairwise independence does not imply independence.
    Say we flip a fair coin twice, then $\Omega=\{(0,0),(0,1),(1,0),(1,1)\}$ and $A=\{(0,0),(0,1)\},B=\{(0,0),(1,0)\},C=\{(0,1),(1,0)\}$, then they are pairwisely independent but not independent.\\
    2. If $A$ is independent of $B$, then $A$ is also independence of $B^c$.
\end{remark}
\begin{definition}
    Let $A,B$ be events such that $\mathbb P(B)>0$.
    The conditional probability is defined as
    $$\mathbb P(A|B)=\frac{\mathbb P(A\cap B)}{\mathbb P(B)}$$
\end{definition}
In particular, $\mathbb P(A|B)=\mathbb P(A)$ iff $A,B$ are independent.
\begin{proposition}
    Let $(A_n)_{n\in\mathbb N}$ be a sequence of disjoint events, then
    $$\mathbb P\left( \bigcup_{n\in\mathbb N}A_n\middle|B \right)=\sum_{n\in\mathbb N}\mathbb P(A_n|B)$$
\end{proposition}
\begin{proof}
    Trivial.
\end{proof}
So conditional probability measure is also a probability measure.
\begin{theorem}[Law of Total Probability]
    Suppose $(B_n)_{n\in\mathbb N}$ is a disjoint sequence of events such that $\mathbb P(B_n)>0$ for all $n$ and $\bigcup B_n=\Omega$.
    Then
    $$\mathbb P(A)=\sum_{n\in\mathbb N}\mathbb P(A|B_n)\mathbb P(B_n)$$
\end{theorem}
\begin{proof}
    Obvious.
\end{proof}
\begin{theorem}[Bayes' Formula]
    Let $(B_n)_{n\in\mathbb N}$ be as above and $\mathbb P(A)>0$, then
    $$\mathbb P(B_n|A)=\frac{\mathbb P(A|B_n)\mathbb P(B_n)}{\sum_{k}\mathbb P(A|B_k)\mathbb P(B_k)}$$
\end{theorem}
\begin{example}[False Positives for a Rare Condition]
    Consider a rare medical condition $A$ which affects $0.1\%$ of the population and a medical test which is positive for $98\%$ of the people affected and $1\%$ of those unaffected by the condition.
    Take a random person, then we want to know the probablity that he has condition $A$ given that he was tested positive.
    Let $A$ be the event of individuals suffering from the condition and $P$ be the individuals being tested positive.
    Hence
    $$\mathbb P(A|P)=\frac{\mathbb P(P|A)\mathbb P(A)}{\mathbb P(P|A^c)\mathbb P(A^c)+\mathbb P(P|A)\mathbb P(A)}\approx 8.9\%$$
    This may seems counterintuitive since this probability seems low, but the thing is that $\mathbb P(P|A^c)>>\mathbb P(P|A)$ since the medical condition is so rare.
\end{example}
\begin{example}[Extra Knowledge Changes Probability]
    Consider the following statements:\\
    (a) I have $2$ childrens, the elder of whom is a boy.\\
    (b) I have $2$ childrens, one of whom is a boy.\\
    (c) I have $2$ childrens, one of whom is a boy who is born on a Tuesday.\\
    We assume equally likely distributions.
    Let $GB$ denotes that the younger is a girl and the elder is a boy.
    Similar for $BG,BB,GG$.
    $$\mathbb P(BB|a)=\mathbb P(BB|BG\cup BB)=\frac{1}{2}$$
    $$\mathbb P(BB|b)=\mathbb P(BB|BG\cup GB\cup BB)=\frac{1}{3}$$
    Write $TN$ be that there are two boy, the younger is born on Tuesday but the elder is not, similar for ($2$-)combinations of $T,N,G$.
    \begin{align*}
        \mathbb P(BB|c)&=\mathbb P(TT\cup TN\cup NT|TT\cup TN\cup NT\cup TG\cup GT)\\
        &=\frac{\mathbb P(TT\cup TN\cup NT)}{\mathbb P(TT\cup TN\cup NT\cup GT\cup TG)}\\
        &=\frac{13}{27}
    \end{align*}
\end{example}
\begin{example}[Simpson's Paradox]
    There are $50$ men and $50$ women applying to a college.
    \begin{center}
        \begin{tabular}{c|c|c|c}
            \hline
            All applicants&Admitted&Rejected&Success Rate\\
            \hline
            State&25&25&50\%\\
            Indep&28&22&56\%\\
            \hline
            Men Only&Admitted&Rejected&Success Rate\\
            \hline
            State&15&22&41\%\\
            Indep&5&8&38\%\\
            \hline
            Women Only&Admitted&Rejected&Success Rate\\
            \hline
            State&10&3&77\%\\
            Indep&23&14&68\%
        \end{tabular}
    \end{center}
    So both men and women in state schools have higher acceptance rate, but the overall acceptance rate of state schools is still lower than that in independent schools.
    This is because the overall acceptance rate for women in this set of data is larger and there are much more of them in independent schools than state schools.
    Basically it is because $A/B>a/b$ and $C/D>c/d$ does not imply $(A+C)/(B+D)>(a+c)/(b+d)$.
\end{example}
