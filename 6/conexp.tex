\section{Conditional Expectation}
\begin{definition}
    Let $B$ be an event such that $\mathbb P(B)>0$ and let $X$ be a random variable, then we define
    $$\mathbb E[X|B]=\frac{\mathbb E[X\cdot 1_B]}{\mathbb P(B)}$$
\end{definition}
\begin{proposition}[Law of Total Expectation]
    Let $\Omega_n$ be a sequence of events that partitions $\Omega$, then
    $$\mathbb E[X]=\sum_n\mathbb E[X|\Omega_n]\mathbb P(\Omega_n)$$
\end{proposition}
\begin{proof}
    Write $X=\sum_nX\cdot1_{\Omega_n}$.
\end{proof}
\begin{definition}
    Let $X_1,\ldots,X_n$ be random variables, then the joint distribution is $\mathbb P(X_1=x_1,\ldots,X_n=x_n),x_i\in\omega_{X_i}$.
    Given such a joint distribution, the marginal distribution of $X_i$ is defined as
    $$\mathbb P(X_i=x_i)=\sum_{j\neq i,x_j\in\Omega_{X_j}}\mathbb P(X_1=x_1,\ldots,X_n=x_n)$$
\end{definition}
\begin{definition}
    Let $X,Y$ be random variables, then the conditional probability given $Y=y,y\in\Omega_Y$ is defined by
    $$\mathbb P(X=x|Y=y)=\mathbb P(X=x,Y=y)/\mathbb P(Y=y)$$
\end{definition}
Immediately
$$\mathbb P(X=x)=\sum_{y\in\Omega_Y}\mathbb P(X=x,Y=y)=\sum_{y\in\Omega_Y}\mathbb P(X=x|Y=y)\mathbb P(Y=y)$$
Note that if $X,Y$ are independent, then
\begin{align*}
    \mathbb P(X+Y=z)&=\sum_{y\in\Omega_Y}\mathbb P(X+Y=z|Y=y)\mathbb P(Y=y)\\
    &=\sum_{y\in\Omega_Y}\mathbb P(X=z-y|Y=y)\mathbb P(Y=y)\\
    &=\sum_{y\in\Omega_Y}\mathbb P(X=z-y)\mathbb P(Y=y)
\end{align*}
Similarly
$$\mathbb P(X+Y=z)=\sum_{x\in\Omega_X}\mathbb P(X=x)\mathbb P(Y=z-x)$$
\begin{example}
    Let $X\sim\operatorname{Pois}(\lambda),Y\sim\operatorname{Pois}(\mu)$, and $X,Y$ are independent, then
    $$\mathbb P(X+Y=n)=\sum_{r=0}^\infty \mathbb P(X=r)\mathbb P(X=n-r)=e^{-(\lambda+\mu)}\frac{(\lambda+\mu)^n}{n!}$$
\end{example}
\begin{definition}
    Let $X,Y$ be random variables,
    The conditional expectation of $X$ given $Y=y$ is defined to be the expectation of the conditional distribution of $X$ given $Y=y$, so
    $$\mathbb E[X|Y=y]=\frac{\mathbb E[X\cdot 1_{Y=y}]}{\mathbb P(Y=y)}=\sum_{x\in\Omega_X}x\mathbb P(X=x|Y=y)$$
\end{definition}
Note that $g(y)=\mathbb E[X|Y=y]$ is a function in $\Omega_Y\to\mathbb R$.
\begin{definition}
    The conditional expectation of $X$ given $Y$ is defined to be $\mathbb E[X|Y]=g(Y)$ which is a random variable as a function of $Y$.
\end{definition}
So obviously,
\begin{align*}
    \mathbb E[X|Y]&=g(Y)\\
    &=\sum_{y\in\Omega_Y}g(Y)\mathbb P(1_{Y=y})\\
    &=\sum_{y\in\Omega_Y}1_{Y=y}g(y)\\
    &=\sum_{y\in\Omega_Y}1_{Y=y}\mathbb E[X|Y=y]
\end{align*}
\begin{example}
    Toss a $p$-coin several times independently, so they produce a sequence of random variables $(X_i)_{i\in\mathbb N}\sim\operatorname{Bern}(p)$ and the number of heads are distributed in $Y_m=X_1+X_2+\cdots+X_m\sim\operatorname{Bin}(m,p)$
    We want to calculate $\mathbb E[X_i|Y_m]$.
    Note that $\mathbb E[X_i|Y_m=r]=\mathbb P(X_i=1|Y_m=r)=r/m$, which means
    $$\mathbb E[X_i|Y_m]=\frac{Y_m}{m}$$
\end{example}
\begin{proposition}
    1. $\forall c\in\mathbb R,\mathbb E[cX|Y]=c\mathbb E[X|Y],\mathbb E[c|Y]=c$.\\
    2.
    $$\mathbb E\left[\sum_{i=1}^nX_i\middle|Y\right]=\sum_{i=1}^n\mathbb E[X_i|Y]$$
    3. $\mathbb E[\mathbb E[X|Y]]=\mathbb E[X]$.
\end{proposition}
\begin{proof}
    1 and 2 are obvious.
    For 3,
    \begin{align*}
        \mathbb E[\mathbb E[X|Y]]&=\sum_{y\in\Omega_Y}\mathbb P(Y=y)\mathbb E[X|Y=y]\\
        &=\sum_{y\in\Omega_Y}\sum_{x\in\Omega_X}x\mathbb P(X=x|Y=y)\mathbb P(Y=y)\\
        &=\sum_{x\in\Omega_X}x\mathbb P(X=x)\\
        &=\mathbb E[X]
    \end{align*}
    As desired.
\end{proof}
Intuitively we also have the following:
\begin{proposition}\label{conditional_exp}
    1. If $X,Y$ are independent, then $\mathbb E[X|Y]=\mathbb E[X]$ which is constant.\\
    2. Suppose $Y,Z$ are independent, then $\mathbb E[\mathbb E[X|Y]|Z]=\mathbb E[X]$.\\
    3. Let $h:\mathbb R\to\mathbb R$, then $\mathbb E[h(Y)X|Y]=h(Y)\mathbb E[X|Y]$.
\end{proposition}
\begin{proof}
    1 is trivial.\\
    For 2, we have $\mathbb E[X|Y]$ is independent of $Z$, so $\mathbb E[\mathbb E[X|Y]|Z]=\mathbb E[\mathbb E[X|Y]]=\mathbb E[X]$ by 1.\\
    As for 3, we have $\mathbb E[h(Y)X|Y=y]=h(y)\mathbb E[X|Y=y]$, the identity follows.
\end{proof}
\begin{corollary}
    $\mathbb E[\mathbb E[X|Y]|Y]=\mathbb E[X|Y]$
\end{corollary}
\begin{proof}
    Take $h(y)=\mathbb E[X|Y=y]$ in Proposition \ref{conditional_exp}.3.
\end{proof}